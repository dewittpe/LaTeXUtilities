% DO NOT COMPILE THIS FILE DIRECTLY!
% This is included by the other .tex files.

\begin{frame}[t,plain]
\titlepage
\end{frame}

\section{Introduction}
\begin{frame}[t]{Brand Identity Standards}
   The \textbf{\color{CUBlue}University of Colorado Denver Brand Identity Standards} provides direction for messaging, design and content of university print and online publications, stationary, signage and other expressions.
   \vskip1em
   Adhering to these standards will help ensure that communications from every university group maintain a clear and uniform message that best represents the image and brand of our two campuses and the University of Colorado as a whole.
\end{frame}
% --------------------------------------------------- Slide --

\section{Our Brand}
\subsection{Overview}
\begin{frame}[t]{Our Brand}
   Our brand is much more than a name or a logo. Every point of contact we have
   with our audience---students, faculty, staff, alumni, donors, and the
   world---builds upon the perception of who we are as a university, the things
   we do to fulfill our mission, and why we are important to our stakeholders.
   \vskip1em
   Our brand serves as our North Star for the decisions we make and the way we
   communicate with our external and internal community. It is our promise, our
   story, our driving force. It is what the world experiences when they
   interact with us, and the benchmark with which we measure our success.
   \vskip1em
   In short, our brand is what makes our university unique.
\end{frame}
% --------------------------------------------------- Slide --

\subsection{Brand Platform}
\begin{frame}[t]{Brand Position}
   The University of Colorado Denver | Anschutz Medical Campus is a dynamic
   university -- comprehensive in scope, entrepreneurial in spirit, and
   innovative at heart. We believe in forward movement -- the promise of
   opportunity, the power of creativity, the impact of discovery. From the
   heart of downtown Denver to our new state-of-the-art health sciences campus,
   we are uniquely positioned to impact the region, the nation and the world.
   \vskip1em
   Our tradition is in the untraditional: Our students contribute to and thrive in
   a diverse cultural, professional and experiential setting; our faculty and
   researchers elevate the definitions of collaboration, inquiry and patient care;
   our alumni and supporters engage with us in fulfilling the shared vision of a
   better world.
\end{frame}
% --------------------------------------------------- Slide --

\section{Theme Usage}
\begin{frame}[t,fragile]
    \frametitle{How to use the theme}
    %\begin{frame}[t,fragile]{How to use the theme}
    \begin{itemize}
        \item Install Beamer
            \begin{itemize}
                \item Some distros have a \verb!latex-beamer! package
            \end{itemize}
        \item Read the Beamer documentation
            \begin{itemize}
                \item \verb!/usr/share/doc/latex-beamer/beameruserguide.pdf.gz! if you are
                    using Debian
                \item \verb!doc/beameruserguide.pdf! in the source package
            \end{itemize}
        \item Install the theme
            \begin{itemize}
                \item If using Linux
                    \begin{itemize}
                        \item \verb!mkdir -p ~/texmf/tex/latex/beamer/UCDenver!\\
                        \item \verb!mv *.sty ~/texmf/tex/latex/beamer/UCDenver!
                        \item \verb!texhash!
                    \end{itemize}
                \item If using TexLive on Mac OS X
                    \begin{itemize}
                        \item \verb!mkdir -p ~/Library/texmf/tex/latex/beamer/UCDenver!\\
                        \item \verb!mv *.sty ~/Library/texmf/tex/latex/beamer/UCDenver!
                        \item \verb!texhash!
                    \end{itemize}
            \end{itemize}
    \end{itemize}
\end{frame}
% --------------------------------------------------- Slide --

\begin{frame}[t,fragile]
    \frametitle{How to use the theme}
    %\begin{frame}[t,fragile]{How to use the theme}
    \begin{itemize}
        \item Changing from black to white (or vice versa)
            \begin{itemize}
                \item Within the preamble you can specify either the black theme or the white
                    theme by changing the logical variable \verb!useblack!.
                    \begin{itemize}
                        \item \verb!\useblacktrue! : black background for title page header with correct logo and white on black footer
                        \item \verb!\useblackfalse! : white background for title page header with correct logo and black on white footer
                    \end{itemize}
            \end{itemize}
    \end{itemize}
\end{frame}
% --------------------------------------------------- Slide --

\subsection{Files}
\begin{frame}[t,fragile]
    \frametitle{Theme files}
\begin{itemize}
\item Themes are composed by sub-themes for single features
\item Inner themes define how the title page, the bullet lists, margins,
      etc. work
  \begin{itemize}
    \item \verb!beamerinnerthemefancy.sty!
  \end{itemize}
\item Outer themes define how headers and footers look like
  \begin{itemize}
    \item \verb!beamerouterthemedecolines.sty!
  \end{itemize}
\item The color theme defines the colors to be used in outer and inner themes
  \begin{itemize}
    \item \verb!beamercolorthemeucdblack.sty!: black and gold footers and headers
    \item \verb!beamercolorthemeucdwhite.sty!: white and gold footers and headers
  \end{itemize}
\item Global themes just include inner, outer and color themes
  \begin{itemize}
    \item \verb!beamerthemeUCDenver.sty!
  \end{itemize}
  \item This is a heavily modified theme based on the Torino theme by Marco Barisione
\end{itemize}
\end{frame}
% --------------------------------------------------- Slide --

\subsection{Configuration}
\begin{frame}[t,fragile]
    \frametitle{Configuring the theme}
   \begin{itemize}
      \item This beamer theme can be configured with options between \verb![! and
         \verb!]!
         \begin{itemize}
            \item \verb!\usetheme[option1 = value, option2 = value]{UCDenver}!
         \end{itemize}
      \item Valid options are: 
         \begin{itemize}
            \item \verb!pageofpages! 
            \item \verb!bullet!
            \item \verb!titleline! 
            \item \verb!alternativetitlepage!
            \item \verb!titlepagelogo!
            \item \verb!headerlogo!
            \item \verb!watermark!
            \item \verb!watermarkheight!
            \item \verb!watermarkheightmult!
         \end{itemize}
      \item If you do not specify any option, you get
         \begin{itemize}
            \item Simple title page
            \item No watermark or logo
            \item Black color theme
            \item Squares for bullet lists
         \end{itemize}
   \end{itemize}
\end{frame}
% --------------------------------------------------- Slide --
\begin{frame}[t,fragile]
    \frametitle{Configuring the theme}
   \begin{itemize}
      \item A logo, shown in the upper right corner, can be choosen with the
         \verb!\logo! command
         \begin{itemize}
            \item \verb!\logo{\includegraphics[height=50px]{logo-image}}!
         \end{itemize}
      \item The official colors have been defined as: \verb!CUGold, CULightGray, CUDarkGray, CUBlue!
   \end{itemize}
\end{frame}
% --------------------------------------------------- Slide --

\begin{frame}[t,fragile]
    \frametitle{Alternative title page}
\begin{itemize}
\item A fancy title page can be enabled with the \verb!alternativetitlepage!
      option
\item You can put a logo in the title page, just pass the file name using the
      \verb!titlepagelogo! option
\item Remember to use a plain and top-aligned frame when using alternative title
      pages:\\
      \vskip1ex
      \verb!\begin{frame}[t,plain]!\\
      \verb!\titlepage!\\
      \verb!\end{frame}!
\end{itemize}
\end{frame}
% --------------------------------------------------- Slide --

\subsection{Watermark}
\begin{frame}[t,fragile]
    \frametitle{Watermark}
\begin{itemize}
\item A watermark can be shown in the bottom right corner of frames
\item Use the \verb!watermark! option to set name of the image file
\item The \verb!watermarkheight! option specifies the height of the watermark
      image
\item It's a good idea to have a big image and shrink it, so it looks good
      when the slide is full screen
\item If the image height in the slide is not the same as the original one,
      you have to use the \verb!watermarkheightmult! option
  \begin{itemize}
  \item For example, if the image is 400 pixel tall but you want it to
        occupy only 100 pixels, use
        \verb![watermarkheight=100px, watermarkheightmult=4]!
  \item It's ugly but I don't know how to fix it
  \end{itemize}
\end{itemize}
\end{frame}
% --------------------------------------------------- Slide --

\watermarkoff
\begin{frame}[t,fragile]
    \frametitle{Disabling the watermark}
\begin{itemize}
\item You may want to disable the watermark on some frames
  \begin{itemize}
  \item For example, an image could partially cover the watermark, with ugly
        results
  \end{itemize}
\item The \verb!\watermarkoff! command can be used to disable the watermark
      in the following frames
\item The \verb!\watermarkon! command restores the watermark in the following
      frames
\item If you did not specify a watermark, nothing happens
\vskip5ex
\item \verb!\watermarkoff! was used for this frame
\end{itemize}
\end{frame}
% --------------------------------------------------- Slide --
\watermarkon

\subsection{Miscellaneous}
\begin{frame}[t,fragile]
    \frametitle{Other options}
\begin{itemize}
\item The \verb!pageofpages! option defines the string between the current
      page number and the total page count
  \begin{itemize}
  \item The default is ``/''
  \item The example files set \verb!pageofpages! to ``of''
  \end{itemize}
\item The \verb!bullet! option can be used to choose the symbol used in
      bullet lists
  \begin{itemize}
  \item \verb!square!: A filled square
        ({\usebeamercolor[fg]{item}\tiny\raise0.2ex\hbox{$\blacksquare$}}) for
        first and third level items, an empty square
        ({\usebeamercolor[fg]{item}\tiny\raise0.2ex\hbox{$\square$}}) for
        second level items
  \item \verb!circle!: A filled circle ({\usebeamercolor[fg]{item}$\bullet$})
        for first and third level items, an empty circle
        ({\usebeamercolor[fg]{item}$\circ$}) for second level items
  \item The default value is \verb!square!
  \end{itemize}
\item If the \verb!titleline! option is set to \verb!true!, a horizontal line
      is drawn below the title
\end{itemize}
\end{frame}
% --------------------------------------------------- Slide --
